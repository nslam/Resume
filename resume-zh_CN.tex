% !TEX TS-program = xelatex
% !TEX encoding = UTF-8 Unicode
% !Mode:: "TeX:UTF-8"

\documentclass{resume}
\usepackage{zh_CN-Adobefonts_external} % Simplified Chinese Support using external fonts (./fonts/zh_CN-Adobe/)
%\usepackage{zh_CN-Adobefonts_internal} % Simplified Chinese Support using system fonts
\usepackage{linespacing_fix} % disable extra space before next section
\usepackage{cite}

\begin{document}
\pagenumbering{gobble} % suppress displaying page number

\name{石力铭}

\basicInfo{
  \email{nslamgg@gmail.com} \textperiodcentered\ 
  \phone{(+86) 180-6737-5826} \textperiodcentered\ 
  \linkedin[liam]{https://www.linkedin.com/in/liam-07a2c/}}
 
\section{\faGraduationCap\  教育背景}
\datedsubsection{\textbf{浙江大学}, 杭州}{2015 -- 至今}
\textit{在读学士}\ 软件工程, 预计 2019 年 6 月毕业

\section{\faUsers\ 实习经历}
\datedsubsection{\textbf{阿里云计算有限公司} 杭州}{2018年4月 -- 2018年10月}
\role{实习}{职位: 基础平台研发工程师}
ECS云盘热迁移
\begin{itemize}
  \item 使用C++实现客户端并发重连方案
  \item 异地备份、迁移达到用户无感知(IO freeze 200ms以下)
\end{itemize}

\datedsubsection{\textbf{Dashbase} 杭州}{2018年10月 -- 2019年4月}
\role{实习}{职位: 软件开发工程师}
Cloud日志流到Elasticsearch集群
\begin{itemize}
  \item 利用Serverless自动化处理云端日志到ES集群中
  \item 支持从对象存储读入,导出到kafka等设置
\end{itemize}

\section{\faGithubAlt\ 个人项目}
\datedsubsection{\textbf{Biolab 生化实验室网站开发}}{2017年4月 -- 2017-8月}
\role{Nginx, Python, Django, Database}{独立后端开发}
\begin{onehalfspacing}
为生化实验室设计开发教学平台
\begin{itemize}
  \item RESTFUL设计
  \item 提供论文、试剂、仪器的资料存储以及检索
  \item 支持群发通知、课程论坛等教学辅助功能
\end{itemize}
\end{onehalfspacing}

\datedsubsection{\textbf{“中控杯”机器人设计与实现}}{2017年2月 -- 2017年5月}
\role{Arduino, C, OpenCV}{机器人比赛项目}
\begin{onehalfspacing}
“中控杯”机器人设计算法
\begin{itemize}
  \item 实现机器人自动寻址
  \item 利用OpenCV中例如SIFT等算法进行目标图像匹配
\end{itemize}
\end{onehalfspacing}

\datedsubsection{\textbf{智能消防栓通讯及管理网站}}{2018 年3月2日 -- 2018年3月4日}
\role{Golang, Python, Django}{独立后端开发}
\begin{onehalfspacing}
智能消防栓管理平台
\begin{itemize}
  \item 使用Golang来处理高并发的消防栓通讯
  \item 使用PostGIS来进行消防栓定位以及查询
\end{itemize}
\end{onehalfspacing}

\datedsubsection{\textbf{Menuapedia设计开发}}{2018 年3月2日 -- 2018年3月4日}
\role{Hackathon in Caltech, America}{团队合作}
\begin{onehalfspacing}
实现了餐馆AR寻址,外文菜单自动翻译、配图功能, \\
视频链接:https://www.youtube.com/watch?v=nBD1JrHceIM&t=3s
\begin{itemize}
  \item 利用iOS ARkit实现餐馆的AR寻址
  \item 利用微软的OCR来识别菜单,同时基于Google的TextAnalysis API分析出菜名,在Google上搜索并爬取图片
\end{itemize}
\end{onehalfspacing}

% Reference Test
%\datedsubsection{\textbf{Paper Title\cite{zaharia2012resilient}}}{May. 2015}
%An xxx optimized for xxx\cite{verma2015large}
%\begin{itemize}
%  \item main contribution
%\end{itemize}

\section{\faCogs\ IT 技能}
% increase linespacing [parsep=0.5ex]
\begin{itemize}[parsep=0.5ex]
  \item \textbf{编程语言: }
  C++ ≈ Python > C ≈ Golang ≈ Java
  \item \textbf{平台: }
  Linux

\end{itemize}

\section{\faHeartO\ 获奖情况}
\datedline{\textit{一等奖}, 浙江省信息学竞赛联赛}{2013 年 11 月}
\datedline{\textit{二等奖}, 浙江大学ACM校赛}{2016 年 4 月}
\datedline{\textit{三等奖}, 浙江大学“中控杯”比赛}{2017 年 5 月}

\section{\faInfo\ 其他}
% increase linespacing [parsep=0.5ex]
\begin{itemize}[parsep=0.5ex]
  \item \textbf{技术博客: }https://nslam.github.io/
  \item \textbf{GitHub: }https://github.com/nslam/
  \item \textbf{语言: }英语 - 熟练(CET-6)
\end{itemize}

%% Reference
%\newpage
%\bibliographystyle{IEEETran}
%\bibliography{mycite}
\end{document}
